% Options for packages loaded elsewhere
\PassOptionsToPackage{unicode}{hyperref}
\PassOptionsToPackage{hyphens}{url}
\PassOptionsToPackage{dvipsnames,svgnames,x11names}{xcolor}
%
\documentclass[
  letterpaper,
  DIV=11,
  numbers=noendperiod]{scrartcl}

\usepackage{amsmath,amssymb}
\usepackage{lmodern}
\usepackage{iftex}
\ifPDFTeX
  \usepackage[T1]{fontenc}
  \usepackage[utf8]{inputenc}
  \usepackage{textcomp} % provide euro and other symbols
\else % if luatex or xetex
  \usepackage{unicode-math}
  \defaultfontfeatures{Scale=MatchLowercase}
  \defaultfontfeatures[\rmfamily]{Ligatures=TeX,Scale=1}
\fi
% Use upquote if available, for straight quotes in verbatim environments
\IfFileExists{upquote.sty}{\usepackage{upquote}}{}
\IfFileExists{microtype.sty}{% use microtype if available
  \usepackage[]{microtype}
  \UseMicrotypeSet[protrusion]{basicmath} % disable protrusion for tt fonts
}{}
\makeatletter
\@ifundefined{KOMAClassName}{% if non-KOMA class
  \IfFileExists{parskip.sty}{%
    \usepackage{parskip}
  }{% else
    \setlength{\parindent}{0pt}
    \setlength{\parskip}{6pt plus 2pt minus 1pt}}
}{% if KOMA class
  \KOMAoptions{parskip=half}}
\makeatother
\usepackage{xcolor}
\usepackage[normalem]{ulem}
\setlength{\emergencystretch}{3em} % prevent overfull lines
\setcounter{secnumdepth}{-\maxdimen} % remove section numbering
% Make \paragraph and \subparagraph free-standing
\ifx\paragraph\undefined\else
  \let\oldparagraph\paragraph
  \renewcommand{\paragraph}[1]{\oldparagraph{#1}\mbox{}}
\fi
\ifx\subparagraph\undefined\else
  \let\oldsubparagraph\subparagraph
  \renewcommand{\subparagraph}[1]{\oldsubparagraph{#1}\mbox{}}
\fi

\usepackage{color}
\usepackage{fancyvrb}
\newcommand{\VerbBar}{|}
\newcommand{\VERB}{\Verb[commandchars=\\\{\}]}
\DefineVerbatimEnvironment{Highlighting}{Verbatim}{commandchars=\\\{\}}
% Add ',fontsize=\small' for more characters per line
\usepackage{framed}
\definecolor{shadecolor}{RGB}{241,243,245}
\newenvironment{Shaded}{\begin{snugshade}}{\end{snugshade}}
\newcommand{\AlertTok}[1]{\textcolor[rgb]{0.68,0.00,0.00}{#1}}
\newcommand{\AnnotationTok}[1]{\textcolor[rgb]{0.37,0.37,0.37}{#1}}
\newcommand{\AttributeTok}[1]{\textcolor[rgb]{0.40,0.45,0.13}{#1}}
\newcommand{\BaseNTok}[1]{\textcolor[rgb]{0.68,0.00,0.00}{#1}}
\newcommand{\BuiltInTok}[1]{\textcolor[rgb]{0.00,0.23,0.31}{#1}}
\newcommand{\CharTok}[1]{\textcolor[rgb]{0.13,0.47,0.30}{#1}}
\newcommand{\CommentTok}[1]{\textcolor[rgb]{0.37,0.37,0.37}{#1}}
\newcommand{\CommentVarTok}[1]{\textcolor[rgb]{0.37,0.37,0.37}{\textit{#1}}}
\newcommand{\ConstantTok}[1]{\textcolor[rgb]{0.56,0.35,0.01}{#1}}
\newcommand{\ControlFlowTok}[1]{\textcolor[rgb]{0.00,0.23,0.31}{#1}}
\newcommand{\DataTypeTok}[1]{\textcolor[rgb]{0.68,0.00,0.00}{#1}}
\newcommand{\DecValTok}[1]{\textcolor[rgb]{0.68,0.00,0.00}{#1}}
\newcommand{\DocumentationTok}[1]{\textcolor[rgb]{0.37,0.37,0.37}{\textit{#1}}}
\newcommand{\ErrorTok}[1]{\textcolor[rgb]{0.68,0.00,0.00}{#1}}
\newcommand{\ExtensionTok}[1]{\textcolor[rgb]{0.00,0.23,0.31}{#1}}
\newcommand{\FloatTok}[1]{\textcolor[rgb]{0.68,0.00,0.00}{#1}}
\newcommand{\FunctionTok}[1]{\textcolor[rgb]{0.28,0.35,0.67}{#1}}
\newcommand{\ImportTok}[1]{\textcolor[rgb]{0.00,0.46,0.62}{#1}}
\newcommand{\InformationTok}[1]{\textcolor[rgb]{0.37,0.37,0.37}{#1}}
\newcommand{\KeywordTok}[1]{\textcolor[rgb]{0.00,0.23,0.31}{#1}}
\newcommand{\NormalTok}[1]{\textcolor[rgb]{0.00,0.23,0.31}{#1}}
\newcommand{\OperatorTok}[1]{\textcolor[rgb]{0.37,0.37,0.37}{#1}}
\newcommand{\OtherTok}[1]{\textcolor[rgb]{0.00,0.23,0.31}{#1}}
\newcommand{\PreprocessorTok}[1]{\textcolor[rgb]{0.68,0.00,0.00}{#1}}
\newcommand{\RegionMarkerTok}[1]{\textcolor[rgb]{0.00,0.23,0.31}{#1}}
\newcommand{\SpecialCharTok}[1]{\textcolor[rgb]{0.37,0.37,0.37}{#1}}
\newcommand{\SpecialStringTok}[1]{\textcolor[rgb]{0.13,0.47,0.30}{#1}}
\newcommand{\StringTok}[1]{\textcolor[rgb]{0.13,0.47,0.30}{#1}}
\newcommand{\VariableTok}[1]{\textcolor[rgb]{0.07,0.07,0.07}{#1}}
\newcommand{\VerbatimStringTok}[1]{\textcolor[rgb]{0.13,0.47,0.30}{#1}}
\newcommand{\WarningTok}[1]{\textcolor[rgb]{0.37,0.37,0.37}{\textit{#1}}}

\providecommand{\tightlist}{%
  \setlength{\itemsep}{0pt}\setlength{\parskip}{0pt}}\usepackage{longtable,booktabs,array}
\usepackage{calc} % for calculating minipage widths
% Correct order of tables after \paragraph or \subparagraph
\usepackage{etoolbox}
\makeatletter
\patchcmd\longtable{\par}{\if@noskipsec\mbox{}\fi\par}{}{}
\makeatother
% Allow footnotes in longtable head/foot
\IfFileExists{footnotehyper.sty}{\usepackage{footnotehyper}}{\usepackage{footnote}}
\makesavenoteenv{longtable}
\usepackage{graphicx}
\makeatletter
\def\maxwidth{\ifdim\Gin@nat@width>\linewidth\linewidth\else\Gin@nat@width\fi}
\def\maxheight{\ifdim\Gin@nat@height>\textheight\textheight\else\Gin@nat@height\fi}
\makeatother
% Scale images if necessary, so that they will not overflow the page
% margins by default, and it is still possible to overwrite the defaults
% using explicit options in \includegraphics[width, height, ...]{}
\setkeys{Gin}{width=\maxwidth,height=\maxheight,keepaspectratio}
% Set default figure placement to htbp
\makeatletter
\def\fps@figure{htbp}
\makeatother

\KOMAoption{captions}{tableheading}
\makeatletter
\makeatother
\makeatletter
\makeatother
\makeatletter
\@ifpackageloaded{caption}{}{\usepackage{caption}}
\AtBeginDocument{%
\ifdefined\contentsname
  \renewcommand*\contentsname{Table of contents}
\else
  \newcommand\contentsname{Table of contents}
\fi
\ifdefined\listfigurename
  \renewcommand*\listfigurename{List of Figures}
\else
  \newcommand\listfigurename{List of Figures}
\fi
\ifdefined\listtablename
  \renewcommand*\listtablename{List of Tables}
\else
  \newcommand\listtablename{List of Tables}
\fi
\ifdefined\figurename
  \renewcommand*\figurename{Figure}
\else
  \newcommand\figurename{Figure}
\fi
\ifdefined\tablename
  \renewcommand*\tablename{Table}
\else
  \newcommand\tablename{Table}
\fi
}
\@ifpackageloaded{float}{}{\usepackage{float}}
\floatstyle{ruled}
\@ifundefined{c@chapter}{\newfloat{codelisting}{h}{lop}}{\newfloat{codelisting}{h}{lop}[chapter]}
\floatname{codelisting}{Listing}
\newcommand*\listoflistings{\listof{codelisting}{List of Listings}}
\makeatother
\makeatletter
\@ifpackageloaded{caption}{}{\usepackage{caption}}
\@ifpackageloaded{subcaption}{}{\usepackage{subcaption}}
\makeatother
\makeatletter
\@ifpackageloaded{tcolorbox}{}{\usepackage[many]{tcolorbox}}
\makeatother
\makeatletter
\@ifundefined{shadecolor}{\definecolor{shadecolor}{rgb}{.97, .97, .97}}
\makeatother
\makeatletter
\makeatother
\ifLuaTeX
  \usepackage{selnolig}  % disable illegal ligatures
\fi
\IfFileExists{bookmark.sty}{\usepackage{bookmark}}{\usepackage{hyperref}}
\IfFileExists{xurl.sty}{\usepackage{xurl}}{} % add URL line breaks if available
\urlstyle{same} % disable monospaced font for URLs
\hypersetup{
  pdftitle={Week 3 Assignment},
  pdfauthor={Emmanel Pascual},
  colorlinks=true,
  linkcolor={blue},
  filecolor={Maroon},
  citecolor={Blue},
  urlcolor={Blue},
  pdfcreator={LaTeX via pandoc}}

\title{Week 3 Assignment}
\author{Emmanel Pascual}
\date{2/21/23}

\begin{document}
\maketitle
\ifdefined\Shaded\renewenvironment{Shaded}{\begin{tcolorbox}[interior hidden, sharp corners, enhanced, breakable, borderline west={3pt}{0pt}{shadecolor}, frame hidden, boxrule=0pt]}{\end{tcolorbox}}\fi

\renewcommand*\contentsname{Table of contents}
{
\hypersetup{linkcolor=}
\setcounter{tocdepth}{2}
\tableofcontents
}
\begin{enumerate}
\def\labelenumi{\arabic{enumi}.}
\tightlist
\item
  Be able to make a new .qmd document Sign top left Create new Quarto
  Document
\end{enumerate}

\begin{itemize}
\item
  Save \textgreater{} Find the posts -\textgreater{} Only use Source
  Editor -\textgreater{}
\item
  Files tab make Quarto Doc it makes it in the folder you're in and
  they're blank
\item
  Top level, it gives reference to what the post is about
\end{itemize}

\begin{enumerate}
\def\labelenumi{\arabic{enumi}.}
\setcounter{enumi}{1}
\tightlist
\item
  Be able to render a .qmd document!
\end{enumerate}

\begin{itemize}
\tightlist
\item
  Yes We can render the whole website too! Render a page from the
  console using commands.
\end{itemize}

\begin{enumerate}
\def\labelenumi{\arabic{enumi}.}
\setcounter{enumi}{2}
\tightlist
\item
  Explain the difference between the source editor view and visual
  editor view in Rstudio
\end{enumerate}

\begin{itemize}
\item
  Source View is plain test (ASCII character will be expressed exactly
  as it is in the source view)
\item
  Visual View Renders the plain text and interprets it as Markdown text,
  it gives the text formatting text instead of the plain test.
\end{itemize}

\begin{enumerate}
\def\labelenumi{\arabic{enumi}.}
\setcounter{enumi}{3}
\tightlist
\item
  Be able to insert simple markdown plain text (headers, lists,
  paragraphs), and render the document.
\end{enumerate}

Headers:

\hypertarget{header-1}{%
\section{Header 1}\label{header-1}}

\hypertarget{header-2}{%
\subsection{Header 2}\label{header-2}}

\hypertarget{header-3}{%
\subsubsection{Header 3}\label{header-3}}

\hypertarget{header-4}{%
\paragraph{Header 4}\label{header-4}}

\hypertarget{header-5}{%
\subparagraph{Header 5}\label{header-5}}

Header 6

The smaller we get the smaller the header will end up being.

Lists:

\emph{-} unordered list + sub-item 1 + sub-item 2 - sub-sub-item 1

\begin{enumerate}
\def\labelenumi{\arabic{enumi}.}
\tightlist
\item
  ordered list
\item
  item 2

  \begin{enumerate}
  \def\labelenumii{\roman{enumii})}
  \tightlist
  \item
    sub-item 1

    \begin{enumerate}
    \def\labelenumiii{\Alph{enumiii}.}
    \tightlist
    \item
      sub-sub-item 1 Paragraphs:
    \end{enumerate}
  \end{enumerate}
\end{enumerate}

\emph{italics} and \textbf{bold}

Plain text End a line with two spaces to start a new paragraph.
\emph{italics} and \emph{italics} \textbf{bold} and \textbf{bold}
superscript\textsuperscript{2} \sout{strikethrough}

Paragraphs:

\begin{enumerate}
\def\labelenumi{\arabic{enumi}.}
\setcounter{enumi}{4}
\tightlist
\item
  Be aware of resources to help you learn more about markdown options.
\end{enumerate}

https://quarto.org/docs/authoring/markdown-basics.html

\begin{enumerate}
\def\labelenumi{\arabic{enumi}.}
\setcounter{enumi}{5}
\tightlist
\item
  Be able to insert an R code chunk, and show the output in the rendered
  document.
\end{enumerate}

\begin{itemize}
\tightlist
\item
  macro for quick insert ctrl - alt - I
\end{itemize}

\begin{Shaded}
\begin{Highlighting}[]
\NormalTok{  summonMe }\OtherTok{\textless{}{-}} \FunctionTok{c}\NormalTok{(}\DecValTok{1}\NormalTok{, }\DecValTok{2}\NormalTok{, }\DecValTok{3}\NormalTok{ , }\DecValTok{4}\SpecialCharTok{:}\DecValTok{6}\NormalTok{)}
\NormalTok{  summonMe}
\end{Highlighting}
\end{Shaded}

\begin{verbatim}
[1] 1 2 3 4 5 6
\end{verbatim}

\begin{enumerate}
\def\labelenumi{\arabic{enumi}.}
\setcounter{enumi}{6}
\tightlist
\item
  Running R code chunks in a qmd
\end{enumerate}

\begin{itemize}
\tightlist
\item
  pressing play
\item
  copy/paste into console
\item
  highlight then command-enter (mac)
\item
  precedence issues (first to last)
\end{itemize}

\begin{Shaded}
\begin{Highlighting}[]
\NormalTok{name }\OtherTok{\textless{}{-}} \StringTok{"Emily"}
\NormalTok{age }\OtherTok{\textless{}{-}} \DecValTok{36}
\NormalTok{today }\OtherTok{\textless{}{-}} \FunctionTok{Sys.Date}\NormalTok{()}
\NormalTok{christmas }\OtherTok{\textless{}{-}} \FunctionTok{as.Date}\NormalTok{(}\StringTok{"2023{-}12{-}25"}\NormalTok{)}
\end{Highlighting}
\end{Shaded}

\begin{enumerate}
\def\labelenumi{\arabic{enumi}.}
\setcounter{enumi}{7}
\tightlist
\item
  Be aware of R code chunk options, and how to use eval, messages,
  error, warning, and echo.
\end{enumerate}

https://quarto.org/docs/computations/execution-options.html

\begin{Shaded}
\begin{Highlighting}[]
\FunctionTok{print}\NormalTok{(}\StringTok{"https://quarto.org/docs/computations/execution{-}options.html}
\StringTok{"}\NormalTok{)}
\end{Highlighting}
\end{Shaded}

\begin{verbatim}
[1] "https://quarto.org/docs/computations/execution-options.html\n"
\end{verbatim}

\begin{enumerate}
\def\labelenumi{\arabic{enumi}.}
\setcounter{enumi}{8}
\item
  Be able to set code chunk options per chunk, and/or for the whole
  document. Understand rules for precedence (which options will apply if
  both are set.)
\item
  Write inline r code.
\end{enumerate}

\begin{Shaded}
\begin{Highlighting}[]
\NormalTok{name }\OtherTok{\textless{}{-}} \StringTok{"Emily"}
\NormalTok{age }\OtherTok{\textless{}{-}} \DecValTok{36}
\NormalTok{today }\OtherTok{\textless{}{-}} \FunctionTok{Sys.Date}\NormalTok{()}
\NormalTok{christmas }\OtherTok{\textless{}{-}} \FunctionTok{as.Date}\NormalTok{(}\StringTok{"2023{-}12{-}25"}\NormalTok{)}
\end{Highlighting}
\end{Shaded}

``My name is Emily and I am 36 years old. It is 267 days until
Christmas, which is my favourite holiday.''

\begin{enumerate}
\def\labelenumi{\arabic{enumi}.}
\setcounter{enumi}{10}
\tightlist
\item
  Explain how the rendering environment is different from the Rstudio
  environment.
\end{enumerate}

\begin{Shaded}
\begin{Highlighting}[]
\FunctionTok{print}\NormalTok{(}\StringTok{"The differences between the rendering eviornment and the Rstudio enviornment are the assignments of the variables. We can either define it outside or inside this specific code chunk."}\NormalTok{)}
\end{Highlighting}
\end{Shaded}

\begin{verbatim}
[1] "The differences between the rendering eviornment and the Rstudio enviornment are the assignments of the variables. We can either define it outside or inside this specific code chunk."
\end{verbatim}

\begin{Shaded}
\begin{Highlighting}[]
\NormalTok{b }\OtherTok{\textless{}{-}} \DecValTok{10}
\NormalTok{c }\OtherTok{\textless{}{-}} \DecValTok{35} 

\NormalTok{a }\OtherTok{\textless{}{-}}\NormalTok{ c}\SpecialCharTok{+}\NormalTok{b }

\NormalTok{a}
\end{Highlighting}
\end{Shaded}

\begin{verbatim}
[1] 45
\end{verbatim}

\begin{enumerate}
\def\labelenumi{\arabic{enumi}.}
\setcounter{enumi}{11}
\tightlist
\item
  Be aware of more advanced quarto options for html documents, and try
  some of the options.
\end{enumerate}



\end{document}
